\section{Discussions}\label{sec:Conclusions}
The use of synthetic MRS data has gained significant interest in recent years due to the need for large datasets with known spectral components for both traditional spectroscopy and deep learning applications. Synthetic data allows researchers to generate unlimited spectra with known ground truth values which has wide-ranging applications, including evaluating the accuracy and precision of data analysis methods. However, the lack of standardization in synthetic data generation poses challenges in terms of reproducibility and generalizability.
 
This work has presented an open-source framework for a modular synthetic data generation model that incorporates well-defined distributions, when available, of physical parameters commonly used in linear-combination modeling software. This framework was intentionally designed so that as research progresses, new information can be added to further improve the realness and accuracy of the simulations. It offers a pathway to generate realistic, in vivo-like synthetic data that captures acquisition-induced artifacts and nuisance signals, and can be applied across a range of MRS applications. This is facilitated by the modular implementation that allows the generation highly tailored datasets for a variety of clinical scenarios. This framework includes a $B_0$ magnetic field simulator that is capable of introducing distortions due to imperfect shimming and high susceptibility effects into the simulations. A novel generator was proposed that is capable of simulating both broad, undulating baseline offsets and highly irregular residual water contributions. Finally, a comprehensive list of spectral components and acquisition-induced artifacts were incorporated. Collectively, these provide a comprehensive, robust, and customizable framework for researchers in need of synthetic data.
 
The development of open-source frameworks for data generation are crucial for ensuring widespread adoption and improving the generalizability of synthetic MRS data and the reproducibility of MRS research. By open-sourcing this work, the authors aim to contribute to the democratization of MRS research by providing access to high quality simulations to a wider community of researchers interested in the field. Furthermore, this will promote greater standardization and reproducibility in the field. Moving forward, collaboration and consensus-building among researchers will be essential to establish standards and best practices for simulating MRS data. Future development can focus on adding additional clinical scenarios to the framework's repertoire, increasing its applicability to even more aspects of MRS. Further research can use this framework to, for example, compare the impact of different spectral modeling components on metabolite quantification or to compare the performance of commonly used spectral fitting models. By addressing the challenges associated with synthetic MRS data generation and promoting standardization, MRS can continue to advance in every aspect of the field. 



