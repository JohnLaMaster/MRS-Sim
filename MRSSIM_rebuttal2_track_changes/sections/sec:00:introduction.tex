\section{Introduction}\label{sec:Introduction}
Magnetic resonance spectroscopy (MRS) is a non-invasive imaging modality that provides in vivo information on the metabolic profile of tissues. It allows for evaluating various pathologies throughout the entire body by analyzing the spectra to quantify metabolite concentrations. In the brain, MRS has been extensively used to investigate a wide range of pathologies, including neurodevelopmental\cite{Augustine2008,Laccetta2022,Tomiyasu2022} and neurodegenerative diseases\cite{Gao2014,Martin2007,McKiernan2023,Oz2016}, inborn errors of metabolism\cite{Cecil2006,Gropman2020,Lai2022}, brain tumors\cite{Calvar2005,Lukas2004,Padelli2022,Nelson2003}, as well as age-related changes\cite{Forester2010,Inglese2004,Reyngoudt2012}. 
 
In order for MRS to be a widely used clinical imaging modality, it is important to ensure the precision and accuracy of MRS data processing and analysis. However, it can be challenging due to variations across scenarios, such as pulse sequence, acquisition parameters, data quality regimes, and biological differences (adult versus pediatric subjects\cite{Dezortova2008} and disease related metabolic patterns\cite{Li2008}). Such comprehensive collections of data are usually not widely available for method development and evaluation due to data privacy restrictions. Furthermore, in vivo data do not have access to ground truth values, making it difficult to determine the reliability, sensitivity, and robustness of different MRS analysis protocols. Phantom data can be useful in developing and validating acquisition and analysis methods, but it usually does not adequately reflect the complexities of in vivo spectra. 

A logical solution to these challenges is to generate synthetic data for a variety of experimental scenarios. However, generating synthetic MRS data can be challenging as well, as it requires adequately incorporating all physical phenomena underlying in vivo data. One of the significant aspects is to choose adequate model parameters that reflect different physiological and pathological conditions. Arguably more important is the need to capture acquisition-induced artifacts and nuisance signals commonly found in in vivo data that are difficult to reproduce in a phantom, such as effects of susceptibility and field inhomogeneity, macromolecule, lipid, and residual water signals, and various other artifacts. Many research groups now routinely use synthetic data, but the data generation models and parameter distributions are rarely made publicly available. This lack of transparency hampers systematic comparison and the formation of consensus best practices for synthetic data generation. 

It is evident from the literature that there is no standard set of best practices for simulating synthetic data. Simulations have been carried out using physical models with a range of complexity and spectral components\cite{Hatami,Das2018a,Das2018,Iqbal2018a}. Most of these methods begin with simulating basis functions, assuming appropriate pulse sequence parameters for their specific scenario. These functions represent metabolites that can be collected and are then modulated by scaling factors representing their underlying concentrations. A simple Lorentzian lineshape is generally applied\cite{Hatami, Das2018a, Das2018} with optional phase offsets\cite{Das2018,Iqbal2018a} and frequency shifts\cite{Hatami}. Finally, some type of broad baseline is typically added. These models are simple and do not fully capture the complexity of in vivo data. Additionally, they often contain non-public components such as baselines, macromolecules, and lipid signals that are either simulated in-house or extracted from private datasets, thus making detailed analysis of their models difficult.

There are several software packages that claim to simulate in vivo spectra, such as FID-A\cite{Simpson2017}, GAVA\cite{Soher2007}, and VESPA\cite{Soher2023}. However, the fundamental difference between these methods and the current work is the definition of "in vivo spectra". Previously, this term referred to simulating idealized spectra to evaluate parameter settings for developing new in vivo pulse sequences. In this work, "in vivo spectra" refers to the data that comes off a clinical scanner prior to any post-processing or analysis. This new definition is meant to reframe the task of simulating synthetic data, to challenge what are assumed to be acceptable simplifications, and to advance the use of synthetic data for MRS applications. 
While pulse sequences are necessary for generating basis sets, they are left to the many software options already available. The most relevant previous work is the recent update for the MARSS\cite{Landheer2021} software which has added a new functionality called synMARSS that simulates synthetic spectra and incorporates many spectral artifacts. They provide an extensive user manual with equations describing their implementation, however, their source code is not available, meaning it cannot be evaluated or modified like MRS-Sim.

Over the past few years, there has been a significant rise in the use of synthetic MRS data. This increase is primarily due to the emergence of machine learning\cite{Das2017} (ML) and deep learning\cite{Gurbani2019, Hatami, Lee2019} (DL) analysis algorithms that require vast amounts of data for training and validation that few centers worldwide possess. Therefore, generating realistic, in vivo-like synthetic data is necessary to facilitate more research into these applications. Synthetic data has known ground truth values that are essential to evaluate the accuracy and precision of the analysis methods being developed. However, growing interest in training ML and DL models with synthetic data will compound the reproducibility and generalizability problems already being experienced in the traditional MRS field.\cite{Craven2022,Marjanska2021,Marjanska2022,Zollner2021} To address this challenge, it is essential to establish standards and best practices for simulating synthetic MRS data. This will require researchers to collaborate and build consensus, as well as develop and adopt open-source frameworks for data generation.


This work presents MRS-Sim: a framework for a modular synthetic data generation model to address some of these challenges. The basic simulation model uses well-defined distributions of physical parameters commonly used in linear-combination modeling software, corresponding to amplitudes, lineshapes, phases, and frequency shifts. Furthermore, it incorporates a realistic $B_0$ map generator to simulate in vivo-like field heterogeneity conditions, and uses semi-parameterized models to describe residual water and smooth background signal contributions. MRS-Sim is designed to simulate spectra at any stage of acquisition from individual coil elements to unaligned transients to fully processed spectra. These features allow researchers to compare new data processing methods, not just modeling algorithms. And the implemented code base is highly flexible and can simulate custom-tailored datasets for a large variety of clinical scenarios. This framework includes a detailed compilation of simulation parameters as well as tools to analyze existing in vivo datasets and extract parameter distributions, which is useful for augmenting in vivo data with similar synthetic data. The open-source code base allows for seamless incorporation of future additions to expand the framework's capabilities to include more types of spectra, spectral model components, and parameter distributions reflecting pathological metabolic signatures. This work is intended to be a community resource to provide researchers, trainees, and experts alike with access to high-quality and comparable synthetic data as a source of continuity in the field.

