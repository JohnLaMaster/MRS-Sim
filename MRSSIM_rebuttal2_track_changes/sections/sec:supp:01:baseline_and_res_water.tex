\section{Baseline and Residual Water Simulations}\label{baseline-simulations}
A single simulation protocol is used to generate the contributions for both baseline offsets and residual water contributions. The capability of generating signal contributions with such widely varying profiles is entirely dependent on the underlying parameters. In order to better understand the capabilities of the proposed generator and its behavior, the following section will explore the effects the various parameters have on the final signal contributions that are generated. For a detailed explanation of how these sampled parameters are converted into their final outputs, please refer to Alg. \ref{alg:smoothed bounded pseudo-random walk} in Sec. \ref{sec:methods:subsubsec:baseline}.

This generator uses a smoothed, pseudo-random bounded walk algorithm for the simulations. For simplicity, this will henceforth be referred to as a \emph{(random) bounded walk}, or simply a \emph{walk}. By using different configuration dictionaries, it is possible to switch from very smooth, undulating baselines to rather erratic and rough residual water regions. The following sections highlight the effects different parameters have on the simulations. 

\subsection{Variables}\label{variables}

These simulations use at least 9 degrees of freedom with two additional optional inputs: 
\begin{enumerate}
    \item Starting height (start)
    \item Ending height (end)
    \item Standard deviation of walk (std)
    \item Lower bound (lower)
    \item Upper bound (upper)
    \item Point density (pt\_density)
    \item PPM range (ppm\_range)
    \item Length of smoothing window (window)
    \item Scale (scale)
    \item Dropout probability (drop\_prob)
    \item *Prime (prime)
\end{enumerate}
Floats or integers can be provided for the point density, prime, and drop\_prob. Everything else should be provided as a list with either a single value or a range in the form of [min\_range, max\_range]. When a single value is provided, that variable will be fixed. When a range is specified, then values will be sampled from that range uniformly. The point density and the PPM range are used to calculate the length of the walks. This length coupled with \(std\) control the flexibility of the raw simulations. The length of the smoothing window, \(window\), the magnitude of the \(std\), and scaling factor, \(scale\), of the baseline will affect the smoothness of the resulting walk. It is specified as a fraction of the length of the walk, so that it is independent of walk length, and can also be either fixed or sampled. \(scale\) determines how prominent each offset will be when added to the simulated data. \(drop\_prob\) is used to randomly omit the offset from that percentage of the simulations. \(prime\) [units: ppm] is used for generating the residual water regions. This allows the region to vary in length by sampling two values in the range of [\(-prime\),\(prime\)) that can expand, contract, or slightly offset the residual water region.

\subsection{Post-processing}\label{post-processing}
Once the random walks have been generated, they are then smoothed with kernels of either fixed or varying width. At this point, trend lines are calculated between the starting and ending points which are then removed so that they start and end on the x-axis. Afterwards, the walks are normalized to [0,1] and then scaled down according to \(scale\). A function called \emph{sim2acquired} then uses zero padding and a nonuniform interpolator to add tails to both sides of the walks so that they match the PPM range of the basis set regarding the spectral width and carrier frequency. Before being added to the FIDs, they are multiplied by the maximum value of the FID in the frequency domain to scale them up to the correct order of magnitude. This makes \(scale\) relative to the maximum height in the spectra. The offsets are added to the FIDs in the frequency domain before an inverse FFT returns the data to the time domain.

The parameters presented below control the length of the walks, their trend lines, flexibility, and smoothness. The default values presented in the configuration dictionaries were determined through preliminary experiments to closely approximate what was observed in a clinical dataset.

\clearpage
\subsection{Preparing for the simulations}\label{preparing-for-the-simulations}
The following section defines parameters for the spectroscopy scenario and the configuration dictionary for the simulations. \\

    \begin{tcolorbox}[breakable, size=fbox, boxrule=1pt, pad at break*=1mm,colback=cellbackground, colframe=cellborder]
\begin{Verbatim}[commandchars=\\\{\}]
\PY{c+c1}{\PYZsh{} Spectroscopy scenario}
\PY{n}{spectralwidth} \PY{o}{=} \PY{l+m+mi}{2000}                       \PY{c+c1}{\PYZsh{} Hz}
\PY{n}{Ns}            \PY{o}{=} \PY{l+m+mi}{2048}                       \PY{c+c1}{\PYZsh{} number of spectral points}
\PY{n}{B0}            \PY{o}{=} \PY{l+m+mf}{3.0}                        \PY{c+c1}{\PYZsh{} T}
\PY{n}{gamma\PYZus{}H}       \PY{o}{=} \PY{l+m+mf}{42.577478518}               \PY{c+c1}{\PYZsh{} MHz/T}
\PY{n}{ppm\PYZus{}ref}       \PY{o}{=} \PY{l+m+mf}{4.65}                       \PY{c+c1}{\PYZsh{} ppm}

\PY{n}{carrier\PYZus{}frequency} \PY{o}{=} \PY{n}{B0} \PY{o}{*} \PY{n}{gamma\PYZus{}H}           \PY{c+c1}{\PYZsh{} MHz}
\PY{n}{ppm} \PY{o}{=} \PY{n}{torch}\PY{o}{.}\PY{n}{linspace}\PY{p}{(}\PY{o}{\PYZhy{}}\PY{l+m+mf}{0.5}\PY{o}{*}\PY{n}{spectralwidth}\PY{p}{,}   
                     \PY{l+m+mf}{0.5}\PY{o}{*}\PY{n}{spectralwidth}\PY{p}{,}\PY{n}{Ns}\PY{p}{)} \PY{c+c1}{\PYZsh{} Hz}
\PY{n}{ppm} \PY{o}{/}\PY{o}{=} \PY{n}{carrier\PYZus{}frequency}                   \PY{c+c1}{\PYZsh{} ppm}
\PY{n}{ppm} \PY{o}{+}\PY{o}{=} \PY{n}{ppm\PYZus{}ref}
\PY{n}{t} \PY{o}{=} \PY{n}{torch}\PY{o}{.}\PY{n}{linspace}\PY{p}{(}\PY{l+m+mi}{0}\PY{p}{,}\PY{n}{Ns}\PY{o}{/}\PY{n}{spectralwidth}\PY{p}{,}\PY{n}{Ns}\PY{p}{)}

\PY{n}{num\PYZus{}samples} \PY{o}{=} \PY{l+m+mi}{10}
\end{Verbatim}
\end{tcolorbox}


\subsection{Understanding the Simulation Plots}
The plots below contain either one or two lines that are blue or red. The default color scheme is as follows:
\begin{itemize}[labelindent=2cm, widest=blue, leftmargin=*, align=left]
    \item[\textbf{\textcolor{pltblue}{blue}}] raw random walk simulation
    \item[\textbf{\textcolor{pltred}{red}}] smoothed walk that is added to the simulated spectra
\end{itemize}
In Sections \ref{baseline:incorporating-the-offsets-into-the-spectra} and \ref{reswater:incorporating-the-offsets-into-the-spectra}, which discuss how to incorporate the walks into the simulated spectra, only a single blue line is plotted. This line is the smoothed, real component of the walk. In Sections \ref{baseline:effect-of-the-hilbert-transform} and \ref{reswater:effect-of-the-hilbert-transform} which show the effect of applying the Hilbert transform, the blue and red lines represent the real and imaginary components, respectively. Similarly, Sections \ref{baseline:compiled-generator} and \ref{reswater:compiled-generator}, which include examples of random simulations for each contribution type, also show the real and imaginary components using blue and red, respectively.

\newpage
\subsection{Baseline Offsets}\label{baseline-offsets}
The following section will explore this generator using the baseline configuration dictionary defined below. The values selected for the plots in the following sections were tailored specifically to the baseline simulations.\\

    \begin{tcolorbox}[breakable, size=fbox, boxrule=1pt, pad at break*=1mm,colback=cellbackground, colframe=cellborder]

\begin{Verbatim}[commandchars=\\\{\}]
\PY{n}{baseline\PYZus{}cfg} \PY{o}{=} \PY{p}{\PYZob{}}
        \PY{l+s+s2}{\PYZdq{}}\PY{l+s+s2}{start}\PY{l+s+s2}{\PYZdq{}}\PY{p}{:}           \PY{p}{[}    \PY{o}{\PYZhy{}}\PY{l+m+mi}{1}\PY{p}{,}     \PY{l+m+mi}{1}\PY{p}{]}\PY{p}{,}
        \PY{l+s+s2}{\PYZdq{}}\PY{l+s+s2}{end}\PY{l+s+s2}{\PYZdq{}}\PY{p}{:}             \PY{p}{[}    \PY{o}{\PYZhy{}}\PY{l+m+mi}{1}\PY{p}{,}     \PY{l+m+mi}{1}\PY{p}{]}\PY{p}{,}
        \PY{l+s+s2}{\PYZdq{}}\PY{l+s+s2}{upper}\PY{l+s+s2}{\PYZdq{}}\PY{p}{:}           \PY{p}{[}            \PY{l+m+mi}{1}\PY{p}{]}\PY{p}{,}
        \PY{l+s+s2}{\PYZdq{}}\PY{l+s+s2}{lower}\PY{l+s+s2}{\PYZdq{}}\PY{p}{:}           \PY{p}{[}           \PY{o}{\PYZhy{}}\PY{l+m+mi}{1}\PY{p}{]}\PY{p}{,}
        \PY{l+s+s2}{\PYZdq{}}\PY{l+s+s2}{std}\PY{l+s+s2}{\PYZdq{}}\PY{p}{:}             \PY{p}{[}  \PY{l+m+mf}{0.05}\PY{p}{,}  \PY{l+m+mf}{0.20}\PY{p}{]}\PY{p}{,}
        \PY{l+s+s2}{\PYZdq{}}\PY{l+s+s2}{window}\PY{l+s+s2}{\PYZdq{}}\PY{p}{:}          \PY{p}{[}  \PY{l+m+mf}{0.15}\PY{p}{,}   \PY{l+m+mf}{0.3}\PY{p}{]}\PY{p}{,}
        \PY{l+s+s2}{\PYZdq{}}\PY{l+s+s2}{pt\PYZus{}density}\PY{l+s+s2}{\PYZdq{}}\PY{p}{:}          \PY{l+m+mi}{128}\PY{p}{,}
        \PY{l+s+s2}{\PYZdq{}}\PY{l+s+s2}{ppm\PYZus{}range}\PY{l+s+s2}{\PYZdq{}}\PY{p}{:}       \PY{p}{[}  \PY{o}{\PYZhy{}}\PY{l+m+mf}{1.6}\PY{p}{,}   \PY{l+m+mf}{8.5}\PY{p}{]}\PY{p}{,}
        \PY{l+s+s2}{\PYZdq{}}\PY{l+s+s2}{scale}\PY{l+s+s2}{\PYZdq{}}\PY{p}{:}           \PY{p}{[}     \PY{l+m+mi}{0}\PY{p}{,}   \PY{l+m+mf}{1}\PY{p}{]}\PY{p}{,}
        \PY{l+s+s2}{\PYZdq{}}\PY{l+s+s2}{drop\PYZus{}prob}\PY{l+s+s2}{\PYZdq{}}\PY{p}{:}           \PY{l+m+mf}{0.0}
\PY{p}{\PYZcb{}}
\end{Verbatim}
\end{tcolorbox}
\clearpage
\subsubsection{Standard Deviation}\label{baseline:standard-deviation}
This section shows the effect of different \(std\) values on the random walk. The \(std\) variable is used when sampling the noise for the walk. It directly controls the amount of variation between two consecutive points prior to the cumulative summation.

\begin{center}
    % \newpage 
Standard Deviation :: STD = 0.05; Window length = random; Kernel\_size = random; Point density = 128


    \begin{center}
    \adjustimage{max size={0.775\linewidth}{0.775\paperheight}}{images/S2:Baseline_exploration/output_14_1.png}
    \end{center}

    
    \newpage

Standard Deviation :: STD = 0.10; Window length = random; Kernel\_size = random; Point density = 128


    \begin{center}
    \adjustimage{max size={0.775\linewidth}{0.775\paperheight}}{images/S2:Baseline_exploration/output_14_3.png}
    \end{center}

    
    \newpage 
Standard Deviation :: STD = 0.15; Window length = random; Kernel\_size = random; Point density = 128


    \begin{center}
    \adjustimage{max size={0.775\linewidth}{0.775\paperheight}}{images/S2:Baseline_exploration/output_14_5.png}
    \end{center}

    
    \newpage 
Standard Deviation :: STD = 0.20; Window length = random; Kernel\_size = random; Point density = 128


    \begin{center}
    \adjustimage{max size={0.775\linewidth}{0.775\paperheight}}{images/S2:Baseline_exploration/output_14_7.png}
    \end{center}

\end{center}
    
\newpage
\subsubsection{Smoothing Kernel}\label{baseline:smoothing-kernel}
In this section, the effects of different smoothing kernel lengths are explored given a fixed \(std\) value.

\begin{center}
Smoothing Kernel :: STD = 0.10; Window Length = 0.10, Kernel\_size = 60; Point density = 128


    \begin{center}
    \adjustimage{max size={0.775\linewidth}{0.775\paperheight}}{images/S2:Baseline_exploration/output_16_1.png}
    \end{center}

    
    \newpage 
Smoothing Kernel :: STD = 0.10; Window Length = 0.20, Kernel\_size = 120; Point density = 128


    \begin{center}
    \adjustimage{max size={0.775\linewidth}{0.775\paperheight}}{images/S2:Baseline_exploration/output_16_3.png}
    \end{center}

    
    \newpage 
Smoothing Kernel :: STD = 0.10; Window Length = 0.30, Kernel\_size = 180; Point density = 128


    \begin{center}
    \adjustimage{max size={0.775\linewidth}{0.775\paperheight}}{images/S2:Baseline_exploration/output_16_5.png}
    \end{center}

\end{center}
    
\newpage
\subsubsection{Point Density}\label{baseline:point-density}
This section explores the effect of varying the point density of the random walk while keeping the \(std\) fixed. Results are presented using two different kernel sizes for the smoothing.

\begin{center}
Point Density :: STD = 0.10, Window Length = [0.10, 0.30], Kernel\_size = [3,9]; Point density = 64


    \begin{center}
    \adjustimage{max size={0.775\linewidth}{0.775\paperheight}}{images/S2:Baseline_exploration/output_18_1.png}
    \end{center}

    
    \newpage 
Point Density :: STD = 0.10, Window Length = [0.10, 0.30], Kernel\_size = [3,9]; Point density = 128


    \begin{center}
    \adjustimage{max size={0.775\linewidth}{0.775\paperheight}}{images/S2:Baseline_exploration/output_18_3.png}
    \end{center}

    
    \newpage 
Point Density :: STD = 0.10, Window Length = [0.10, 0.30], Kernel\_size = [3,9]; Point density = 256


    \begin{center}
    \adjustimage{max size={0.775\linewidth}{0.775\paperheight}}{images/S2:Baseline_exploration/output_18_5.png}
    \end{center}

\end{center}
    
\newpage
\subsubsection{Incorporating the offsets into the spectra}\label{baseline:incorporating-the-offsets-into-the-spectra}
To increase variability, the starting and ending heights are randomly selected. When considering the entire spectrum, however, there are two options to avoid unrealistic, hard transition points. These variables can be manually adjusted or the trend line can be removed, which is how the current code works.

\begin{center}
With the original trend lines still included


    \begin{center}
    \adjustimage{max size={0.775\linewidth}{0.775\paperheight}}{images/S2:Baseline_exploration/output_20_1.png}
    \end{center}

    
    \newpage 
With the original trend lines removed


    \begin{center}
    \adjustimage{max size={0.775\linewidth}{0.775\paperheight}}{images/S2:Baseline_exploration/output_20_3.png}
    \end{center}
    { \hspace*{\fill} \\}
\end{center}

\newpage
\subsubsection{Effect of the Hilbert Transform}\label{baseline:effect-of-the-hilbert-transform}
Simulating spectra requires complex spectral components, including the baseline and residual water. The Hilbert transform is used to generate those corresponding imaginary components. 

    \begin{center}
    \adjustimage{max size={0.775\linewidth}{0.775\paperheight}}{images/S2:Baseline_exploration/output_22_0.png}
    \end{center}


\newpage
\subsubsection{Compiled Generator}\label{baseline:compiled-generator}
This shows the variety of baselines that can be generated when randomly sampling all variables.
    \begin{center}
    \adjustimage{max size={0.775\linewidth}{0.775\paperheight}}{images/S2:Baseline_exploration/output_26_0.png}
    \end{center}

\newpage
\subsection{Residual Water}\label{residual-water}

The following section will explore this generator using the residual water dictionary defined below. While the sections are the same as in Sec. \ref{baseline-offsets}, the values displayed are tailored to the ranges for the residual water simulations.

\subsubsection{Configuration dictionaries}\label{configuration-dictionaries}
The following dictionary entries are the standard defaults in MRS-Sim for generating residual water contributions.\\

    \begin{tcolorbox}[breakable, size=fbox, boxrule=1pt, pad at break*=1mm,colback=cellbackground, colframe=cellborder]
\begin{Verbatim}[commandchars=\\\{\}]
\PY{n}{resWater\PYZus{}cfg} \PY{o}{=} \PY{p}{\PYZob{}}
        \PY{l+s+s2}{\PYZdq{}}\PY{l+s+s2}{start}\PY{l+s+s2}{\PYZdq{}}\PY{p}{:}           \PY{p}{[}            \PY{l+m+mi}{0}\PY{p}{]}\PY{p}{,}
        \PY{l+s+s2}{\PYZdq{}}\PY{l+s+s2}{end}\PY{l+s+s2}{\PYZdq{}}\PY{p}{:}             \PY{p}{[}            \PY{l+m+mi}{0}\PY{p}{]}\PY{p}{,}
        \PY{l+s+s2}{\PYZdq{}}\PY{l+s+s2}{upper}\PY{l+s+s2}{\PYZdq{}}\PY{p}{:}           \PY{p}{[}     \PY{l+m+mi}{0}\PY{p}{,}     \PY{l+m+mi}{1}\PY{p}{]}\PY{p}{,}
        \PY{l+s+s2}{\PYZdq{}}\PY{l+s+s2}{lower}\PY{l+s+s2}{\PYZdq{}}\PY{p}{:}           \PY{p}{[}     \PY{l+m+mi}{0}\PY{p}{,}     \PY{l+m+mi}{1}\PY{p}{]}\PY{p}{,}
        \PY{l+s+s2}{\PYZdq{}}\PY{l+s+s2}{std}\PY{l+s+s2}{\PYZdq{}}\PY{p}{:}             \PY{p}{[}   \PY{l+m+mf}{0.2}\PY{p}{,}  \PY{l+m+mf}{0.40}\PY{p}{]}\PY{p}{,}
        \PY{l+s+s2}{\PYZdq{}}\PY{l+s+s2}{window}\PY{l+s+s2}{\PYZdq{}}\PY{p}{:}          \PY{p}{[}  \PY{l+m+mf}{0.05}\PY{p}{,}  \PY{l+m+mf}{0.15}\PY{p}{]}\PY{p}{,}
        \PY{l+s+s2}{\PYZdq{}}\PY{l+s+s2}{pt\PYZus{}density}\PY{l+s+s2}{\PYZdq{}}\PY{p}{:}         \PY{l+m+mi}{1204}\PY{p}{,}
        \PY{l+s+s2}{\PYZdq{}}\PY{l+s+s2}{ppm\PYZus{}range}\PY{l+s+s2}{\PYZdq{}}\PY{p}{:}       \PY{p}{[}   \PY{l+m+mf}{4.4}\PY{p}{,}   \PY{l+m+mf}{4.9}\PY{p}{]}\PY{p}{,}
        \PY{l+s+s2}{\PYZdq{}}\PY{l+s+s2}{prime}\PY{l+s+s2}{\PYZdq{}}\PY{p}{:}              \PY{l+m+mf}{0.15}\PY{p}{,}
        \PY{l+s+s2}{\PYZdq{}}\PY{l+s+s2}{scale}\PY{l+s+s2}{\PYZdq{}}\PY{p}{:}           \PY{p}{[}   \PY{l+m+mf}{1.0}\PY{p}{,}   \PY{l+m+mf}{1.0}\PY{p}{]}\PY{p}{,} \PY{c+c1}{\PYZsh{} typically: [0.05, 0.20] - better visualization.}
        \PY{l+s+s2}{\PYZdq{}}\PY{l+s+s2}{drop\PYZus{}prob}\PY{l+s+s2}{\PYZdq{}}\PY{p}{:}           \PY{l+m+mf}{0.0}
\PY{p}{\PYZcb{}}
\end{Verbatim}
\end{tcolorbox}

\clearpage
\subsubsection{Standard Deviation}\label{standard-deviation}%\raggedright
This section shows the effect of different \(std\) values on the random walk. The \(std\) variable is used when sampling the noise for the walk. It directly controls the amount of variation between two consecutive points prior to the cumulative summation.

    \begin{center} 
Standard Deviation :: STD = 0.20; Window length = random; Kernel\_size = random; Point density = 1204


    \begin{center}
    \adjustimage{max size={0.775\linewidth}{0.775\paperheight}}{images/S2:Baseline_exploration/output_30_1.png}
    \end{center}
    % { \hspace*{\fill} \\}
    
    \newpage 
Standard Deviation :: STD = 0.30; Window length = random; Kernel\_size = random; Point density = 1204

    \begin{center}
    \adjustimage{max size={0.775\linewidth}{0.775\paperheight}}{images/S2:Baseline_exploration/output_30_3.png}
    \end{center}
    { \hspace*{\fill} \\}
    
    \newpage 
Standard Deviation :: STD = 0.40; Window length = random; Kernel\_size = random; Point density = 1204


    \begin{center}
    \adjustimage{max size={0.775\linewidth}{0.775\paperheight}}{images/S2:Baseline_exploration/output_30_5.png}
    \end{center}
    { \hspace*{\fill} \\}
    
    \newpage 
Standard Deviation :: STD = 0.50; Window length = random; Kernel\_size = random; Point density = 1204


    \begin{center}
    \adjustimage{max size={0.775\linewidth}{0.775\paperheight}}{images/S2:Baseline_exploration/output_30_7.png}
    \end{center}
    { \hspace*{\fill} \\}
\end{center}

\clearpage
\subsubsection{Smoothing Kernel}\label{reswater:smoothing-kernel}%\raggedright
In this section, the effects of different smoothing kernel lengths are explored given a fixed \(std\) value.

\begin{center}
Smoothing Kernel :: STD = 0.10, Window Length = 0.05, Kernel\_size = 30; Point density = 1204


    \begin{center}
    \adjustimage{max size={0.775\linewidth}{0.775\paperheight}}{images/S2:Baseline_exploration/output_32_1.png}
    \end{center}

    
    \newpage 
Smoothing Kernel :: STD = 0.10, Window Length = 0.10, Kernel\_size = 60; Point density = 1204


    \begin{center}
    \adjustimage{max size={0.775\linewidth}{0.775\paperheight}}{images/S2:Baseline_exploration/output_32_3.png}
    \end{center}
    { \hspace*{\fill} \\}
    
    \newpage 
Smoothing Kernel :: STD = 0.10, Window Length = 0.15, Kernel\_size = 90; Point density = 1204



    \begin{center}
    \adjustimage{max size={0.775\linewidth}{0.775\paperheight}}{images/S2:Baseline_exploration/output_32_5.png}
    \end{center}
    { \hspace*{\fill} \\}
\end{center}

\clearpage
\subsubsection{Point Density}\label{reswater:point-density}%\raggedright
This section explores the effect of varying the point density of the random walk while keeping the \(std\) fixed. Results are presented using two different kernel sizes for the smoothing.

    
\begin{center}
Point Density :: STD = 0.10, Window Length = [0.05, 0.15], Kernel\_size = [12,38]; Point density = 512


    \begin{center}
    \adjustimage{max size={0.775\linewidth}{0.775\paperheight}}{images/S2:Baseline_exploration/output_34_1.png}
    \end{center}
    
    \newpage 
Point Density :: STD = 0.10, Window Length = [0.05, 0.15], Kernel\_size = [12,38]; Point density = 1024


    \begin{center}
    \adjustimage{max size={0.775\linewidth}{0.775\paperheight}}{images/S2:Baseline_exploration/output_34_3.png}
    \end{center}
    { \hspace*{\fill} \\}
    
    \newpage 
Point Density :: STD = 0.10, Window Length = [0.05, 0.15], Kernel\_size = [12,38]; Point density = 2048


    \begin{center}
    \adjustimage{max size={0.775\linewidth}{0.775\paperheight}}{images/S2:Baseline_exploration/output_34_5.png}
    \end{center}
    { \hspace*{\fill} \\}
\end{center}

\clearpage
\subsubsection{Incorporating the offsets into the spectra}\label{reswater:incorporating-the-offsets-into-the-spectra}%\raggedright
To increase variability, the starting and ending heights are randomly selected. When considering the entire spectrum, however, these must be adjusted to avoid unrealistic, hard transition points. As mentioned above, this is achieved by removing the trend lines.

    % \begin{Verbatim}[commandchars=\\\{\}]
\begin{center}
With the original trend lines still included


    \begin{center}
    \adjustimage{max size={0.775\linewidth}{0.775\paperheight}}{images/S2:Baseline_exploration/output_36_1.png}
    \end{center}

    
\newpage
With the original trend lines removed


    \begin{center}
    \adjustimage{max size={0.775\linewidth}{0.775\paperheight}}{images/S2:Baseline_exploration/output_36_3.png}
    \end{center}

\end{center}

\clearpage
\subsubsection{Effect of the Hilbert Transform}\label{reswater:effect-of-the-hilbert-transform}%\raggedright
Simulating spectra requires complex spectral components, including the baseline and residual water. The Hilbert transform is used to generate those corresponding imaginary components. 

    \begin{center}
    \adjustimage{max size={0.775\linewidth}{0.775\paperheight}}{images/S2:Baseline_exploration/output_38_0.png}
    \end{center}


\clearpage
\subsubsection{Compiled Generator}\label{reswater:compiled-generator}%\raggedright
This shows the variety of residual water contributions that can be generated when randomly sampling all variables.

    \begin{center}
    \adjustimage{max size={0.775\linewidth}{0.775\paperheight}}{images/S2:Baseline_exploration/output_42_0.png}
    \end{center}
    % { \hspace*{\fill} \\}