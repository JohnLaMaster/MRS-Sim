\section{Discussion}\label{sec:discussion}
This work presents a flexible, open-source framework to help researchers generate complex, in vivo-like MRS data. Spectral components from state-of-the-art spectral fitting models were combined to create a comprehensive model. Flexibility and modularity are key components that make this framework a viable option for simulating in vivo-like data representing a multitude of clinical scenarios.

The use of synthetic MRS data has become increasingly popular in recent years due to the need for large datasets with known spectral components. Using synthetic data allows researchers to create unlimited spectra with known ground truth values. This is useful for both traditional spectroscopy and deep learning applications, including evaluating the accuracy and precision of data analysis methods. However, there is a lack of standardization in synthetic data generation, which poses challenges in terms of reproducibility and generalizability.

Researchers currently use a variety of simulation frameworks to generate synthetic data for their particular needs. These frameworks often involve simplifying assumptions regarding the spectral components, which excludes some components that are deemed non-essential for that particular task. These assumptions make synthetic data less generalizable, similar to phantom data which struggles to approximate in vivo data. In general, simulating data has two primary challenges: which spectral components and artifacts to include and what parameter values to use for the included components. These choices can be difficult for experts, let alone non-expert users. Moreover, the heterogeneity in the simulation methods used makes reproducibility challenging.
 
MRS-Sim addresses these challenges as an open-source solution for a synthetic data generation model that includes a comprehensive list of spectral components and acquisition-induced artifacts. Several novel contributions are presented to include experimental artifacts that are usually ignored during spectral modeling. These modules are designed to make the simulations more comprehensive and in vivo-like. The first contribution is a $B_0$ field simulator. It is capable of introducing distortions due to imperfect shimming and high susceptibility effects into the simulations. Then, a semi-parametric generator is proposed that can simulate both broad, undulating baseline offsets and highly irregular residual water contributions. The implementation of the presented framework is flexible and can be updated with new information as research progresses to improve the realness and accuracy of the simulations. 

The modularity of the MRS-Sim implementation is one of its key features, allowing for flexible and customizable use of the framework. Firstly, the functions that add artifacts and spectral components are modular and independent, which makes it very easy to understand the code and its purpose. Secondly, it allows for the easy generation of highly customized datasets for various clinical scenarios, as the simulation step itself is also modular. Consequently, this feature enables easy community contributions to develop new functionalities, thus preventing the need for entirely new frameworks. New scenarios, such as J-difference edited spectra, diffusion spectra, or 2D spectra, can be developed and seamlessly integrated.

The appendix and digital repository include a table that provides up-to-date information about moiety- and metabolite-level characterizations of things like spin-systems, temperature-induced artifacts, $T_2$ values, and metabolite concentration ranges. This information can be continually updated to provide the community with comprehensive and state-of-the-art information about brain metabolites at a glance. Additionally, these suggested values provide a good starting point for simulating datasets. Cumulatively, this framework offers a pathway to generate realistic, in vivo-like synthetic data that captures acquisition-induced artifacts and nuisance signals and can be applied across a range of MRS applications. 