\begin{refsection}
\section{B\textsubscript{0} Field Simulator}
\begin{figure}[h!]
    \centering
    \includegraphics[width=0.5\textwidth]{images/S2:Baseline_exploration/SI_B0_field_model.png}
    \caption{$B_0$ field volume with labels corresponding to the mathematical model for simulating the heterogeneities.}
    \label{fig:SI_B0 model}
\end{figure}

As described in Li et al. \cite{Li2015}, using the $B_0$ field map, measured during the acquisition, to modulate the basis functions prior to fitting showed significant improvement in metabolite quantification and the fit residual surrounding the metabolite peaks. The fact that this technique has a direct impact on metabolite quantification shows the importance of including it when simulating spectra. The first step is to define the spectral and imaging resolutions of the simulated acquisitions. The defaults, shown below, assume a cubic spectroscopy voxel, but any cuboidal shape is acceptable.

\begin{itemize}[labelindent=2cm, widest=spectroscopy voxel, leftmargin=*, align=left] 
    \item[\textbf{spectroscopy voxel}] \textnormal{[10 mm x  10 mm x 10 mm]}
    \item[\textbf{imaging voxel}] \textnormal{[0.5mm x 0.5mm x 0.5mm]}
\end{itemize}

The dimensions of the volume being modeled are the same as the spectroscopy voxel. The number of points included in the simulation is the quotient of the spectroscopy resolution and the anatomical imaging resolution.
\begin{equation}
    \mathbf{num\_pts} = \frac{\textnormal{spectroscopy resolution}}{\textnormal{imaging resolution}}
    = \left[{\frac{\textnormal{10mm}}{\textnormal{0.5mm}}}_X, {\frac{\textnormal{10mm}}{\textnormal{0.5mm}}}_Y, {\frac{\textnormal{10mm}}{\textnormal{0.5mm}}}_Z\right]
\end{equation}

Once the model volume has been defined, the actual $B_0$ field, $dB0$, can then be modeled. Magnetic fields do not have hard, disjointed heterogeneities, but smooth variations. This is true even in the presence of high susceptibility effects causing strong distortions as long as the voxels are reasonably small. This model therefore assumes linearly varying gradients across the volume of the voxel. This assumption allowed the problem to be reduced to four variables:
\begin{description}[labelindent=2cm]
    \item[$\mathbf{dx}$] half of the mean change in the x-direction
    \item[$\mathbf{dy}$] half of the mean change in the y-direction
    \item[$\mathbf{dz}$] half of the mean change in the z-direction
    \item[$\boldsymbol{\mu\ \ }$] the mean offset of the entire voxel
\end{description}

The next step defines the gradients in each direction as shown in Eqns. \ref{eqn:dB0_x}, \ref{eqn:dB0_y}, and \ref{eqn:dB0_z}. As mentioned above, this model assumes linearly varying gradients, but more complex gradients can be implemented manually at this stage.
\begin{align} 
    &\mathbf{x} = \textnormal{linspace}(\textbf{-dx}, \textbf{dx}, \textbf{num\_pts}_\textnormal{X}) \label{eqn:dB0_x}\\
    &\mathbf{y} = \textnormal{linspace}(\textbf{-dy}, \textbf{dy}, \textbf{num\_pts}_\textnormal{Y}) \label{eqn:dB0_y}\\
    &\mathbf{z} = \textnormal{linspace}(\textbf{-dz}, \textbf{dz}, \textbf{num\_pts}_\textnormal{Z}) \label{eqn:dB0_z}
\end{align}

Once the gradients have been defined, the field map can be modeled very easily, as shown in Eqn. \ref{eqn:calculate dB0}. The primary challenge in this step is reshaping the tensors to get the correct n+3D output.
% \begin{equation*}
\begin{align}
% \begin{split} 
\label{eqn:calculate dB0}
    &\mathbf{dB0} = \textbf{x}*\textbf{y}*\textbf{z} + \boldsymbol{\mu}\textnormal{, such that}\\ &\mathbf{dB0}.size() = torch.Size([batchSize, \ldots, \textbf{num\_pts}_\textnormal{X}, \textbf{num\_pts}_\textnormal{Y}, \textbf{num\_pts}_\textnormal{Z}])
% \end{split}
\end{align}
% \end{equation*}

Modulation due to the heterogeneous magnetic field can be applied in both the time-domain and the frequency-domain. In the time-domain, the complex exponential is applied via multiplication with the basis functions, as is shown below in Eqn. \ref{eqn:SI_B0_modulation}. In the frequency-domain, the complex exponential needs to be convolved with the spectrum. In both mathematical and practical terms, it is simpler to apply the modulation in the time-domain. Therefore, this model has only implemented the time-domain modulation.
\begin{equation}\label{eqn:SI_B0_modulation}
    F(t) = \sum_n^N M_n * basisfcn_n * \underbrace{ \sum_{r=1}^R e^{-i\Delta\omega_r t} }_{B_0\ inhomog.} \textnormal{, where }
    \Delta\omega = \mathbf{dB0}.flatten(dims=[-3:-1]) \textnormal{ and } R=cumprod(\textbf{num\_pts})
\end{equation}
In this equation, the model \textbf{dB0} is flattened along the last three dimensions representing the \textbf{x}-, \textbf{y}-, and \textbf{z}-dimensions respectively. Therefore, \textbf{r} represents the linear indices from [1,R] corresponding the subscripts of the points within the 3D volume. The effective magnetic field at each of those points within the voxel has a direct effect on each moeity that resonates and contributes to the overall signal. Because of this, each basis function is modulated independently before summation. 


% \bibliography{supp_bib_sec_00}
\printbibliography
% \putbib
\end{refsection}