\section{Discussions}\label{sec:Conclusions}
The use of synthetic MRS data has gained significant interest in recent years due to the need for large datasets with known spectral components for both traditional spectroscopy and deep learning applications. Synthetic data allows researchers to generate unlimited spectra with known ground truth values which has wide-ranging applications, including evaluating the accuracy and precision of data analysis methods. However, the lack of standardization in synthetic data generation poses challenges in terms of reproducibility and generalizability.

As discussed in the introduction, researchers currently use a variety of simulation frameworks to generate synthetic data for their particular need. Oftentimes these include simplifying assumptions for the spectral components considered non-essential for their task. These assumptions limit the generalizability of synthetic data in the same way phantom data struggles to approximate in vivo data. In general, simulating data has two primary challenges: what spectral components and artifacts to include, and what parameter values to use for the included components. Such choices can be challenging, even for expert users, making much of this out of reach for non-expert users. In addition to being challenging to simulate, such heterogeneity in the simulation methods complicates reproducibility.%Even expert users can struggle to find appropriate values. 
% And much like simulating a basis set for a custom pulse sequence requires knowledge that is generally unavailable to non-expert users, in vivo-like simulations can also be 

 
To address these issues, this work has presented an open-source framework for a modular synthetic data generation model. %Standard spectral artifacts have been incorporated and can be easily applied. 
A comprehensive list of spectral components and acquisition-induced artifacts were incorporated. 
Several novel contributions are proposed to include experimental artifacts that are traditionally overlooked during spectral modeling, making these simulations comprehensive and in vivo-like. The first is a $B_0$ magnetic field simulator that is capable of introducing distortions due to imperfect shimming and high susceptibility effects into the simulations. Then, a novel generator was proposed that is capable of simulating both broad, undulating baseline offsets and highly irregular residual water contributions. This framework was intentionally designed so that as research progresses, new information can be added to further improve the realness and accuracy of the simulations. To further that point, a table of has been compiled in the appendix and the repository that provides up-to-date moiety- and metabolite-level characterizations of things like spin-systems, temperature-induced artifacts, T2 values, and metabolite concentration ranges. This can be continually updated to provide the community with comprehensive and state-of-the-art information about brain metabolites at a glance. Additionally, these suggested values provide a good starting point for simulating datasets. Cumulatively, this offers a pathway to generate realistic, in vivo-like synthetic data that captures acquisition-induced artifacts and nuisance signals, and can be applied across a range of MRS applications. 

The modularity of MRS-Sim is another key feature of this framework. First and foremost is that modularity of the functions adding artifacts and spectral components makes it very easy to look at the code and understand what it is doing. Secondly, it allows for the easy generation of highly tailored datasets for a variety of clinical scenarios because the simulation step itself is also modular. Consequently, this makes it easy for community contributions to further develop new functionalities. The authors believe this is a crucial aspect %to promote wide-spread adoption 
because it won't leave a gap that needs to be filled by entirely new frameworks. New scenarios, such as J-difference edited spectra, diffusion spectra, or, eventually, 2D spectra, can be developed and seamlessly integrated.
 
The development of open-source frameworks for data generation are crucial for ensuring widespread adoption and improving the generalizability of synthetic MRS data and the reproducibility of MRS research. By open-sourcing this work, the authors aim to contribute to the democratization of MRS research by providing access to high quality information and simulations to a wider community of researchers interested in the field. Improving the quality of synthetic data inherently improves its generalizability to in vivo data with the natural consequence being improved overall applicability of synthetic data in MRS. Furthermore, this will promote greater standardization and reproducibility. Moving forward, collaboration and consensus-building among researchers will be essential to establish standards and best practices for simulating MRS data. As mentioned above, future development can focus on adding additional clinical scenarios to the framework's repertoire, increasing its applicability to even more aspects of MRS. Further research can use this framework to, for example, compare the impact of different spectral modeling components on metabolite quantification or to compare the performance of commonly used spectral fitting models. By addressing the challenges associated with synthetic MRS data generation and promoting standardization, MRS can continue to advance in every aspect of the field. 