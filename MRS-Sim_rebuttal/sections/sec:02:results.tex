\section{Results}\label{sec:Results}
The selection of simulations presented in this section focus on short echo (TE=30ms) 3T PRESS spectra with a spectral width of 2000Hz and randomly sampled parameters. The concentrations were sampled with respect to total creatine (tCr). The choice of short echo spectra is motivated by their ability to capture a broader range of metabolite peaks compared to long echo spectra. The simulations presented include metabolites sampled from a comprehensive set of common brain metabolites including Asc, Asp, Ch, Cr, GABA, Gln, Glu, GPC, GSH, Lac, mI, NAA, NAAG, PCh, PCr, PE, sI, Tau, and a variety of macromolecular and lipid basis functions. For consistency, the SNR was fixed to 15dB and the chemical shift reference point is set to 4.65ppm.

It is important to note that while pulse sequence implications are integral to the overall simulation process, their detailed exploration is beyond the scope of this work. Therefore, in this study, the simulations are specifically tailored to the PRESS sequence. 
% While the pulse sequence implications are acknowledged as an important aspect of MRS simulations, their exploration is beyond the immediate scope of this work. 
The main objective of this work is to manipulate pre-simulated basis sets until they approximate in vivo spectra. Therefore, the primary machinations of MRS-Sim are the same regardless of the basis functions used. Overall, the presented simulations using the PRESS sequence provide a valuable foundation for understanding the performance and capabilities of the MRS-Sim framework.

\subsection{Baseline and Residual Water Generator}
\begin{figure}[t!]
    \centering
    \begin{tabular}[c]{ccc}
    \begin{subfigure}[c]{0.31\textwidth}
        \includegraphics[width=0.93\textwidth,keepaspectratio]{images/30ms_samples/curated/30ms_curated_sample{1}.eps}
        \vspace{3pt}
    \end{subfigure}&
    \begin{subfigure}[c]{0.31\textwidth}
        \includegraphics[width=0.93\textwidth,keepaspectratio]{images/30ms_samples/curated/30ms_curated_sample{2}.eps}
        \vspace{3pt}
    \end{subfigure}&
    \begin{subfigure}[c]{0.31\textwidth}
        \includegraphics[width=0.93\textwidth,keepaspectratio]{images/30ms_samples/curated/30ms_curated_sample{3}.eps}
        \vspace{3pt}
    \end{subfigure}\\
    \begin{subfigure}[c]{0.31\textwidth}
        \includegraphics[width=0.93\textwidth,keepaspectratio]{images/30ms_samples/curated/30ms_curated_sample{4}.eps}
        \vspace{3pt}
    \end{subfigure}&
    \begin{subfigure}[c]{0.31\textwidth}
        \includegraphics[width=0.93\textwidth,keepaspectratio]{images/30ms_samples/curated/30ms_curated_sample{5}.eps}
        \vspace{3pt}
    \end{subfigure}&%
    \begin{subfigure}[c]{0.31\textwidth}
        \includegraphics[width=0.93\textwidth,keepaspectratio]{images/30ms_samples/curated/30ms_curated_sample{6}.eps}
        \vspace{3pt}
    \end{subfigure}\\
    \begin{subfigure}[c]{0.31\textwidth}
        \includegraphics[width=0.93\textwidth,keepaspectratio]{images/30ms_samples/curated/30ms_curated_sample{7}.eps}
        \vspace{3pt}
    \end{subfigure}&
    \begin{subfigure}[c]{0.31\textwidth}
        \includegraphics[width=0.93\textwidth,keepaspectratio]{images/30ms_samples/curated/30ms_curated_sample{8}.eps}
        \vspace{3pt}
    \end{subfigure}&
    \begin{subfigure}[c]{0.31\textwidth}
        \includegraphics[width=0.93\textwidth,keepaspectratio]{images/30ms_samples/curated/30ms_curated_sample{9}.eps}
        \vspace{3pt}
    \end{subfigure}\\
    \begin{subfigure}[c]{0.31\textwidth}
        \includegraphics[width=0.93\textwidth,keepaspectratio]{images/30ms_samples/curated/30ms_curated_sample{10}.eps}
        \vspace{3pt}
    \end{subfigure}&
    \begin{subfigure}[c]{0.31\textwidth}
        \includegraphics[width=0.93\textwidth,keepaspectratio]{images/30ms_samples/curated/30ms_curated_sample{11}.eps}
        \vspace{3pt}
    \end{subfigure}&%
    \begin{subfigure}[c]{0.31\textwidth}
        \includegraphics[width=0.93\textwidth,keepaspectratio]{images/30ms_samples/curated/30ms_curated_sample{12}.eps}
        \vspace{3pt}
    \end{subfigure}\\
    \begin{subfigure}[c]{0.31\textwidth}
        \includegraphics[width=0.93\textwidth,keepaspectratio]{images/30ms_samples/curated/30ms_curated_sample{13}.eps}
    \end{subfigure}&
    \begin{subfigure}[c]{0.31\textwidth}
        \includegraphics[width=0.93\textwidth,keepaspectratio]{images/30ms_samples/curated/30ms_curated_sample{14}.eps}
    \end{subfigure}&%
    \begin{subfigure}[c]{0.31\textwidth}
        \includegraphics[width=0.93\textwidth,keepaspectratio]{images/30ms_samples/curated/30ms_curated_sample{15}.eps}
    \end{subfigure}
    \end{tabular}
    \caption{Sample spectra simulated for a PRESS sequence with TE=30ms that highlight the effect of the baseline and residual water contributions.}
    \label{fig:30ms samples curated clean}
\end{figure}


Fig. \ref{fig:30ms samples curated clean} illustrates a single clean short echo (TE=30ms) PRESS spectrum showcasing various combinations of residual water and baseline contributions. The metabolite concentrations and lineshapes were fixed in addition to the settings described above. These samples omit spectral artifacts to more clearly highlight the variety that can be achieved by randomly sampling parameters for the baseline and residual water generator. 

It should be noted that it is currently unknown what a true baseline looks like. Different baseline modeling protocols produce different baselines and because in vivo data does not have known ground truths, there is no way to identify which method is the most correct. Because of this, the proposed generator produces a wide variety of baselines that approximate the baselines extracted by traditional methods. The motivation behind this approach is the idea that if the generator cannot be limited to strictly in vivo-like baselines, then it should incorporate a variety of appropriate baseline profiles such that true baselines are included. Because this generator is not based on a baseline fitting method, future work could use it to evaluate the baseline modeling performance of various fitting protocols given different baseline profiles.

\subsection{Complete Model}
In contrast, Fig. \ref{fig:30ms samples curated dirty} presents a simulated PRESS spectrum (TE=30ms) with randomly sampled artifacts, and offsets. %The dropout probability for each parameter was set to 50\%. This means that the model parameters, including metabolites, are randomly omitted from the simulations. This helps to emphasize the variety of outputs that are possible for a single set of defined ranges and distributions. 
This figure highlights the variety a single spectrum can assume just by randomly sampling the artifacts. Sophisticated simulations, such as these, are useful for developing and validating data processing techniques such as artifact removal and new spectral fitting protocols. Spectra can be simulated to be as clean or as dirty as desired, all while maintaining known ground truth values. %Because every component in these simulations has a known ground truth, even the most challenging spectra can 



It should be noted that the baselines and residual water regions shown sometimes appear to be uncorrelated with the depicted spectra. This is due to additional artifacts that are applied to the simulations after all of the spectral components are combined. The plotted baseline and residual water regions are the ground truths and do not have additional artifacts applied to them. %Sophisticated simulations, such as these, are useful for developing and validating artifact removal techniques and spectral fitting protocols.  
\begin{figure}[ht!]
    \centering
    \begin{tabular}[c]{ccc}
    \begin{subfigure}[c]{0.31\textwidth}
        \includegraphics[width=0.93\textwidth]{images/30ms_samples/dirty/30ms_curated_dirty_sample{1}.eps}
        \vspace{3pt}
    \end{subfigure}&
    \begin{subfigure}[c]{0.31\textwidth}
        \includegraphics[width=0.93\textwidth]{images/30ms_samples/dirty/30ms_curated_dirty_sample{2}.eps}
        \vspace{3pt}
    \end{subfigure}&
    \begin{subfigure}[c]{0.31\textwidth}
        \includegraphics[width=0.93\textwidth]{images/30ms_samples/dirty/30ms_curated_dirty_sample{3}.eps}
        \vspace{3pt}
    \end{subfigure}\\
    \begin{subfigure}[c]{0.31\textwidth}
        \includegraphics[width=0.93\textwidth]{images/30ms_samples/dirty/30ms_curated_dirty_sample{4}.eps}
        \vspace{3pt}
    \end{subfigure}&
    \begin{subfigure}[c]{0.31\textwidth}
        \includegraphics[width=0.93\textwidth]{images/30ms_samples/dirty/30ms_curated_dirty_sample{5}.eps}
        \vspace{3pt}
    \end{subfigure}&%
    \begin{subfigure}[c]{0.31\textwidth}
        \includegraphics[width=0.93\textwidth]{images/30ms_samples/dirty/30ms_curated_dirty_sample{6}.eps}
        \vspace{3pt}
    \end{subfigure}\\
    \begin{subfigure}[c]{0.31\textwidth}
        \includegraphics[width=0.93\textwidth]{images/30ms_samples/dirty/30ms_curated_dirty_sample{7}.eps}
        \vspace{3pt}
    \end{subfigure}&
    \begin{subfigure}[c]{0.31\textwidth}
        \includegraphics[width=0.93\textwidth]{images/30ms_samples/dirty/30ms_curated_dirty_sample{8}.eps}
        \vspace{3pt}
    \end{subfigure}&
    \begin{subfigure}[c]{0.31\textwidth}
        \includegraphics[width=0.93\textwidth]{images/30ms_samples/dirty/30ms_curated_dirty_sample{9}.eps}
        \vspace{3pt}
    \end{subfigure}\\
    \begin{subfigure}[c]{0.31\textwidth}
        \includegraphics[width=0.93\textwidth]{images/30ms_samples/dirty/30ms_curated_dirty_sample{10}.eps}
        \vspace{3pt}
    \end{subfigure}&
    \begin{subfigure}[c]{0.31\textwidth}
        \includegraphics[width=0.93\textwidth]{images/30ms_samples/dirty/30ms_curated_dirty_sample{11}.eps}
        \vspace{3pt}
    \end{subfigure}&%
    \begin{subfigure}[c]{0.31\textwidth}
        \includegraphics[width=0.93\textwidth]{images/30ms_samples/dirty/30ms_curated_dirty_sample{12}.eps}
        \vspace{3pt}
    \end{subfigure}\\
    \begin{subfigure}[c]{0.31\textwidth}
        \includegraphics[width=0.93\textwidth]{images/30ms_samples/dirty/30ms_curated_dirty_sample{13}.eps}
    \end{subfigure}&
    \begin{subfigure}[c]{0.31\textwidth}
        \includegraphics[width=0.93\textwidth]{images/30ms_samples/dirty/30ms_curated_dirty_sample{14}.eps}
    \end{subfigure}&%
    \begin{subfigure}[c]{0.31\textwidth}
        \includegraphics[width=0.93\textwidth]{images/30ms_samples/dirty/30ms_curated_dirty_sample{15}.eps}
    \end{subfigure}\\
    \end{tabular}
    \caption{Sample spectra, similar to Fig. \ref{fig:30ms samples curated clean} [PRESS, TE=30ms], with randomly sampled, uncorrected artifacts to approximate raw data.}%Sample spectra simulated for a PRESS sequence with TE=30ms. These spectra have sampled parameters just like Fig. \ref{fig:30ms samples curated clean} plus a variety of uncorrected artifacts including eddy currents, zero- and first-order phase offsets, frequency shifts, macromolecule and lipid contributions, residual water, and baseline offsets.}
    \label{fig:30ms samples curated dirty}
\end{figure}



